\documentclass[a4paper,11pt]{article}
\usepackage{amsmath}
\usepackage{amssymb}
\usepackage{mathtools}
\usepackage[cdot,amssymb]{SIunits}
\usepackage{graphicx}
\usepackage{hyperref}
\usepackage{verbatim}
\usepackage{braket}

\allowdisplaybreaks


\author{Jiajun Ren \\ \href{mailto:renjj14@mails.tsinghua.edu.cn}{renjj14@mails.tsinghua.edu.cn}}
\title{Pariser-Parr-Pople-Peierls Model}


\begin{document}
	% generate the title
	\maketitle
    The PPP model
    \begin{align}
        \hat{H} & = \sum_{{i}\sigma} \varepsilon_i a_{i\sigma}^\dagger a_{i\sigma}
        +  \sum_{\{ij\}\sigma} t_{ij} a_{i\sigma}^\dagger a_{j\sigma} \nonumber \\
        & + \sum_{i} U_i n_{i\uparrow} n_{i\downarrow}  
            + \sum_{i<j} V_{ij} (n_i-Z_i) (n_j-Z_j)
    \end{align}
    The longe range potential $V_{ij}$ is calculated with Ohno-Klopper analytical formula:
    \begin{gather}
        \overline{U}=(U_i+U_j)/2 \\
        V_{ij}= \frac{U}{\sqrt{1+U^2 R_{ij}^2/14.397^2}}
    \end{gather}

    The Peierls model only consider the nearest neighour hopping affected by
    linear electron-phonon interaction

    \begin{gather}
        t_{i,i+1}=t_{i,i+1}^0+\alpha \Delta_i \\
        \Delta_i=(u_{i+1}-u_i)
    \end{gather}
    when $\Delta_i>0$, the bond length increase, so the $|t_{ij}|$ decrease, $t$
    is negative. So, $\alpha>0$.

    Nuclear interaction because of phonon (dynamic) 
    \begin{gather}
        H_{nn}^{'}= \frac{1}{2} K \sum_{i} \Delta_i^2
    \end{gather}

    The static nuclear interaction is included in the PPP term $ V_{ij} Z_i Z_j $
    
    The electron-electron interaction changed because of phonon
    \begin{align}
        V_{i,i+1} & =V_{i,i+1}^0+ %
        (\frac{\partial{V_{i,i+1}}}{\partial{R_{i,i+1}}})_{R_{i,i+1}=R_0} %
        \Delta_i \\ \nonumber
        &  = V_{i,i+1}^0-\frac{RU^3/14.397^2}{(1+U^2R^2/14.397^2)^{3/2}} \Delta_i
    \end{align}
    The first order term is defined as $W_{ij}$
    \begin{gather}
        H_{ee}^{'}=-\sum_{i} W_{i,i+1} \Delta_i (n_i-Z_i) (n_i+1-Z_i+1)
    \end{gather}
    There are two components, one is on-site energy, another is electron
    electron energy, and the other is nuclear nuclear energy.  The nuclear
    nuclear part vanish if the system is equal and the length is fixed. It
    should vanish in principle, because the $1/2K \Delta_i^2$ exist. This term
    is linear term if not vanish, the physical meaning?
 
    All the nuclear interaction changed because of the pi electrons. nothing
    about the $\sigma$ electrons.

    The phonon effect on site energy is included in the PPP term.

    Then, the PPPP model is
    \begin{align}
        \hat{H} & = \hat{H}_0 %
        +  \sum_{\{i\}\sigma} \alpha \Delta_i a_{i\sigma}^\dagger a_{i+1\sigma} %
        + \textrm{conjugated} \nonumber \\
        &  - \sum_{i} W_{i,i+1} \Delta_i (n_i-Z_i) (n_{i+1}-Z_{i+1}) \nonumber \\
        &  +\frac{1}{2} K \sum_{i} \Delta_i^2
    \end{align}

    The model is different from optimization in ab initio calculation. Because
    it include electron-phone coupling, it is dynamic not static optimization.

    There is a boudary condition that the displacement total effect is 0.
    \begin{gather}
        \sum_{i} \Delta_i=0 \label{eq:boundary_condition}
    \end{gather}
    Then
    \begin{align}
        L & = E-\eta\sum_{i} \Delta_i \nonumber \\
          & = E_0 +  \sum_{\{i \in bond \}\sigma} \alpha \Delta_i \langle %
        \hat{T}_i \rangle \\ \nonumber
        &  - \sum_{i \in bond } W_{i} \Delta_i  \langle \hat{D}_i \rangle \nonumber \\
        &  +\frac{1}{2} K \sum_{i} \Delta_i^2 -\eta\sum_{i} \Delta_i \nonumber
    \end{align}
    define
    \begin{gather}
        \langle \hat{T}_i \rangle = \langle a_{i\sigma}^\dagger a_{i+1\sigma} + \textrm{conjugated} \rangle\\
        \langle \hat{D}_i \rangle = \langle (n_i-Z_i) (n_{i+1}-Z_{i+1}) \rangle
    \end{gather}

    \begin{gather}
        \frac{\partial{L}}{\Delta_i}= \alpha \langle \hat{T}_i \rangle - W_i %
        \langle \hat{D}_i \rangle + K \Delta_i -\eta = 0 \label{eq:peierls_sc} 
    \end{gather}
    sum Eq.\eqref{eq:peierls_sc} and use the boundary condition
    Eq.\eqref{eq:boundary_condition}
    \begin{gather}
        \eta=\frac{1}{N} \sum_{i \in \textrm{bond} }( \alpha \langle \hat{T}_i \rangle - W_i \langle %
        \hat{D}_i \rangle )
    \end{gather}
    Then
    \begin{gather}
        \Delta_i = \frac {- \alpha \langle \hat{T}_i \rangle + W_i \langle %
        \hat{D}_i \rangle +\eta} {K} \label{eq:delta_i}
    \end{gather} 

    self consistent calculation
    Since the eigenfunction depends on $\Delta_i$, we need guess a initial
    $\Delta_i$ and update it with Eq.\eqref{eq:delta_i} until the energy or
    $\Delta_i$ is not changed. $\Delta_i$ as the criterion is better.
    
    The updated integral should base on the original integral not the last
    updated step. Since we want to get the eigenfunction of the whole
    hamiltnian and the delta on this wavefunction and original hamiltonian.
    

    

\end{document}
